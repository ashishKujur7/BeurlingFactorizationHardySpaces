\documentclass{beamer}

\usepackage[english]{babel}
\usepackage{metalogo}
\usepackage{listings}
\usepackage{fontspec}
\usepackage{tikz}

\newtheorem{claim}[theorem]{Claim}
\newtheorem{proposition}[theorem]{Proposition}
\newtheorem{notation}[theorem]{Notation}
\newtheorem{observation}[theorem]{Observation}
\newtheorem{conjecture}[theorem]{Conjecture}
\newtheorem{exercise}[theorem]{Exercise}
\newtheorem{question}[theorem]{Question}

\numberwithin{equation}{subsection}

\usepackage[backend=biber,style=alphabetic]{biblatex}
\addbibresource{main.bib}

% section symbol
%\renewcommand{\thesection}{\S\arabic{section}}

% \renewcommand{\Pr}{{\bf Pr}}
% \newcommand{\Prx}{\mathop{\bf Pr\/}}
% \newcommand{\E}{{\bf E}}
% \newcommand{\Ex}{\mathop{\bf E\/}}
% \newcommand{\Var}{{\bf Var}}
% \newcommand{\Varx}{\mathop{\bf Var\/}}
% \newcommand{\Cov}{{\bf Cov}}
% \newcommand{\Covx}{\mathop{\bf Cov\/}}

% shortcuts for symbol names that are too long to type
\newcommand{\eps}{\epsilon}
\newcommand{\lam}{\lambda}
\renewcommand{\l}{\ell}
\newcommand{\la}{\langle}
\newcommand{\ra}{\rangle}
\newcommand{\wh}{\widehat}
\newcommand{\wt}{\widetilde}

% % "blackboard-fonted" letters for the reals, naturals etc.
\newcommand{\R}{\mathbb R}
\newcommand{\N}{\mathbb N}
\newcommand{\Z}{\mathbb Z}
\newcommand{\F}{\mathbb F}
\newcommand{\Q}{\mathbb Q}
\newcommand{\C}{\mathbb C}
\newcommand{\D}{\mathbb D}
\newcommand{\T}{\mathbb T}

% % operators that should be typeset in Roman font
% \newcommand{\poly}{\mathrm{poly}}
% \newcommand{\polylog}{\mathrm{polylog}}
% \newcommand{\sgn}{\mathrm{sgn}}
% \newcommand{\avg}{\mathop{\mathrm{avg}}}
% \newcommand{\val}{{\mathrm{val}}}

% % complexity classes
% \renewcommand{\P}{\mathrm{P}}
% \newcommand{\NP}{\mathrm{NP}}
% \newcommand{\BPP}{\mathrm{BPP}}
% \newcommand{\DTIME}{\mathrm{DTIME}}
% \newcommand{\ZPTIME}{\mathrm{ZPTIME}}
% \newcommand{\BPTIME}{\mathrm{BPTIME}}
% \newcommand{\NTIME}{\mathrm{NTIME}}

% values associated to optimization algorithm instances
\newcommand{\Opt}{{\mathsf{Opt}}}
\newcommand{\Alg}{{\mathsf{Alg}}}
\newcommand{\Lp}{{\mathsf{Lp}}}
\newcommand{\Sdp}{{\mathsf{Sdp}}}
\newcommand{\Exp}{{\mathsf{Exp}}}

% if you think the sum and product signs are too big in your math mode; x convention
% as in the probability operators
\newcommand{\littlesum}{{\textstyle \sum}}
\newcommand{\littlesumx}{\mathop{{\textstyle \sum}}}
\newcommand{\littleprod}{{\textstyle \prod}}
\newcommand{\littleprodx}{\mathop{{\textstyle \prod}}}


% calligraphic letters
\newcommand{\calA}{{\cal A}}
\newcommand{\calB}{{\cal B}}
\newcommand{\calC}{{\cal C}}
\newcommand{\calD}{{\cal D}}
\newcommand{\calE}{{\cal E}}
\newcommand{\calF}{{\cal F}}
\newcommand{\calG}{{\cal G}}
\newcommand{\calH}{{\cal H}}
\newcommand{\calI}{{\cal I}}
\newcommand{\calJ}{{\cal J}}
\newcommand{\calK}{{\cal K}}
\newcommand{\calL}{{\cal L}}
\newcommand{\calM}{{\cal M}}
\newcommand{\calN}{{\cal N}}
\newcommand{\calO}{{\cal O}}
\newcommand{\calP}{{\cal P}}
\newcommand{\calQ}{{\cal Q}}
\newcommand{\calR}{{\cal R}}
\newcommand{\calS}{{\cal S}}
\newcommand{\calT}{{\cal T}}
\newcommand{\calU}{{\cal U}}
\newcommand{\calV}{{\cal V}}
\newcommand{\calW}{{\cal W}}
\newcommand{\calX}{{\cal X}}
\newcommand{\calY}{{\cal Y}}
\newcommand{\calZ}{{\cal Z}}

% proof
%\renewcommand{\qedsymbol}{$\ddot\smile$}


% bold letters (useful for random variables)
%----------------------------------------------
% \renewcommand{\a}{{\boldsymbol a}}
% \renewcommand{\b}{{\boldsymbol b}}
% \renewcommand{\c}{{\boldsymbol c}}
% \renewcommand{\d}{{\boldsymbol d}}
% \newcommand{\e}{{\boldsymbol e}}
% \newcommand{\f}{{\boldsymbol f}}
% \newcommand{\g}{{\boldsymbol g}}
% \newcommand{\h}{{\boldsymbol h}}
% \renewcommand{\i}{{\boldsymbol i}}
% \renewcommand{\j}{{\boldsymbol j}}
% \renewcommand{\k}{{\boldsymbol k}}
% \newcommand{\m}{{\boldsymbol m}}
% \newcommand{\n}{{\boldsymbol n}}
% \renewcommand{\o}{{\boldsymbol o}}
% \newcommand{\p}{{\boldsymbol p}}
% \newcommand{\q}{{\boldsymbol q}}
% \renewcommand{\r}{{\boldsymbol r}}
% \newcommand{\s}{{\boldsymbol s}}
% \renewcommand{\t}{{\boldsymbol t}}
% \renewcommand{\u}{{\boldsymbol u}}
% \renewcommand{\v}{{\boldsymbol v}}
% \newcommand{\w}{{\boldsymbol w}}
% \newcommand{\x}{{\boldsymbol x}}
% \newcommand{\y}{{\boldsymbol y}}
% \newcommand{\z}{{\boldsymbol z}}
% \newcommand{\A}{{\boldsymbol A}}
% \newcommand{\B}{{\boldsymbol B}}
% \newcommand{\C}{{\boldsymbol C}}
% \newcommand{\D}{{\boldsymbol D}}
% \newcommand{\E}{{\boldsymbol E}}
% \newcommand{\F}{{\boldsymbol F}}
% \newcommand{\G}{{\boldsymbol G}}
% \renewcommand{\H}{{\boldsymbol H}}
% \newcommand{\I}{{\boldsymbol I}}
% \newcommand{\J}{{\boldsymbol J}}
% \newcommand{\K}{{\boldsymbol K}}
% \renewcommand{\L}{{\boldsymbol L}}
% \newcommand{\M}{{\boldsymbol M}}
% \renewcommand{\O}{{\boldsymbol O}}
% \renewcommand{\P}{{\mathbb{P}}}
% \newcommand{\Q}{{\boldsymbol Q}}
% \newcommand{\R}{{\boldsymbol R}}
% \renewcommand{\S}{{\boldsymbol S}}
% \newcommand{\T}{{\boldsymbol T}}
% \newcommand{\U}{{\boldsymbol U}}
% \newcommand{\V}{{\boldsymbol V}}
% \newcommand{\W}{{\boldsymbol W}}
% \newcommand{\X}{{\boldsymbol X}}
% \newcommand{\Y}{{\boldsymbol Y}}
% \newcommand{\Z}{{\boldsymbol Z}}

% script letters
\newcommand{\scrA}{{\mathscr A}}
\newcommand{\scrB}{{\mathscr B}}
\newcommand{\scrC}{{\mathscr C}}
\newcommand{\scrD}{{\mathscr D}}
\newcommand{\scrE}{{\mathscr E}}
\newcommand{\scrF}{{\mathscr F}}
\newcommand{\scrG}{{\mathscr G}}
\newcommand{\scrH}{{\mathscr H}}
\newcommand{\scrI}{{\mathscr I}}
\newcommand{\scrJ}{{\mathscr J}}
\newcommand{\scrK}{{\mathscr K}}
\newcommand{\scrL}{{\mathscr L}}
\newcommand{\scrM}{{\mathscr M}}
\newcommand{\scrN}{{\mathscr N}}
\newcommand{\scrO}{{\mathscr O}}
\newcommand{\scrP}{{\mathscr P}}
\newcommand{\scrQ}{{\mathscr Q}}
\newcommand{\scrR}{{\mathscr R}}
\newcommand{\scrS}{{\mathscr S}}
\newcommand{\scrT}{{\mathscr T}}
\newcommand{\scrU}{{\mathscr U}}
\newcommand{\scrV}{{\mathscr V}}
\newcommand{\scrW}{{\mathscr W}}
\newcommand{\scrX}{{\mathscr X}}
\newcommand{\scrY}{{\mathscr Y}}
\newcommand{\scrZ}{{\mathscr Z}}

\newcommand{\im}{{\text{im }}}
\newcommand{\ip}[1]{\left\langle #1 \right\rangle}
\newcommand{\norm}[1]{\left\lVert #1 \right\rVert}
\newcommand{\abs}[1]{\left\lvert #1 \right\rvert}


\usetheme{Nord}
% \usetheme[style=light]{Nord}

%\setmainfont{JetBrains Mono Regular}
%\setsansfont{JetBrains Mono Regular}
%\setmonofont{JetBrains Mono Regular}

%\setmainfont{Noto Sans Regular}
%\setsansfont{Noto Sans Regular}
%\setmonofont{Noto Sans Mono Regular}

\AtBeginSection[]
{
  \begin{frame}[c,noframenumbering,plain]
    \tableofcontents[sectionstyle=show/hide,subsectionstyle=show/show/hide]
  \end{frame}
}

\AtBeginSubsection[]
{
  \begin{frame}[c,noframenumbering,plain]
    \tableofcontents[sectionstyle=show/hide,subsectionstyle=show/shaded/hide]
  \end{frame}
}

\title{\texorpdfstring{$H^{p}$}{Hp} spaces}
\subtitle{A Study of \texorpdfstring{$H^{p}$}{Hp}  spaces and Inner Outer Factorization of functions in \texorpdfstring{$H^{p}$}{Hp} spaces}
\author{Ashish Kujur}
\institute{MSC21304, Indian Institute of Science Education and Research, Thiruvananthapuram}
\date{15th May, 2023}

\begin{document}
 \begin{frame}[plain,noframenumbering]
   \maketitle
 \end{frame}
 \section{Notations}
%% Frame 1 %%
 \begin{frame}{Notation and Conventions}
     \begin{itemize}
	     \pause
	 \item (Open Unit Disc) $\D = \left\{ z\in \C : \abs{z} < 1 \right\}$.
	 \item (Unit Circle) $\T = \left\{ z\in \C : \abs{z}=1 \right\}$.
	     \pause
	 \item $\T$ has the normalised Lebesgue measure $\frac{dt}{2\pi}$ unless specified otherwise.
	 \item $\calM \left( \T \right)$: Banach space of complex measures on $\T$ with the total variation norm.
	     \pause
	 \item \textit{nth Fourier coefficient} of $f\in \calL ^{1}\left( \T \right)$ and $\mu \in \calM \left( \T \right)$, $n\in \Z$:
	     \begin{equation*}
		 \hat{f}(n) := \frac{1}{2\pi} \int_{-\pi}^{\pi} f\left( e^{it} \right) e^{-int} dt.
	     \end{equation*}
	     \begin{equation*}
		 \hat{\mu} \left( n \right) := \int_{\T} e^{-int} d\mu \left( e^{it} \right).
	     \end{equation*}
	 \item $H\left( \D \right) = \left\{ f : \D \to \C : f \text{ is holomorphic on } \D \right\}$ \\
	     $h\left( \D \right) = \left\{ f: \D \to \C : f \text{ is harmonic in } \D \right\}$.
     \end{itemize}
 \end{frame}

 \section{Hardy Spaces}
 %% Frame 2 %%
 \begin{frame}{Hardy Spaces on $\D$}{(F. Riesz (1923))}
     \pause
     Let $1\le p \le \infty$ and $f\in H \left( \D \right)$.

For $0\le r < 1$, define $f_{r} : \T \to \C$, $f_{r} \left( e^{i\theta} \right) = f\left( re^{i\theta} \right)$ for each $e^{i\theta} \in \T$.
     \pause

     Define $H^{p} \left( \D \right)$, \textit{Hardy class of analytic functions}, by 
     \begin{equation*}
	 H^{p} \left( \D \right) = \left\{ f\in H\left( \D \right) : \left\{ \norm{f_{r}}_{L^{p} \left( \T \right)} \right\}_{0\le r < 1} \text{ is bounded} \right\} 
     \end{equation*}
\pause
\begin{theorem}[\cite{hardy1915mean}]
	 If $f\in H^{p} \left( \D \right)$ then
	 $\norm{f_{r_{1}}}_{L^{p} \left( \T \right)} \le \norm{f_{r_{2}}}_{L^{p} \left( \T \right)}$
	 for $0 < r_{1} \le r_{2} < 1$.
	 \label{thm:increasing-norm}
     \end{theorem}
\pause
     \begin{theorem}
	 For $1\le p \le \infty$, $H^{p} \left( \D \right)$ is a Banach space with the norm
	 \begin{equation*}
	     \norm{f}_{H^{p} \left( \D \right)}:=\sup_{0< r < 1} \norm{f_r}_{L^{p} \left( \T \right)} = \lim_{r \to 1} \norm{f_{r}}_{L^{p} \left( \T \right)}.
	 \end{equation*}
	 \label{thm:HpD-Banach}
     \end{theorem}
 \end{frame}

%% Frame 2 %%
 \begin{frame}{Hardy Spaces on $\T$}
     \pause
     Let $1 \le p \le \infty$. Consider the measure space $\left( \T, \calB\left( \T \right), dt/2\pi \right)$. 
     Define
     $H^{p} \left( \T \right) = \left\{ f\in L^{p} \left( \T \right) : \hat{f}\left( n \right) = 0 \text{ for each } n<0 \right\}$.
     
     $H^{p}\left( \T \right)$ is a Banach space.

     \pause

     Define $\calM_{a} \left( \T \right) = \left\{ \mu \in \calM \left( \T \right) : \hat{\mu} \left( n \right) = 0 \text{ for each } n < 0 \right\}$.
     $\calM_{a} \left( \T \right)$ is a Banach space.

     \pause
     \begin{question}
	 $H^{p} \left( \D \right)$, $H^{p} \left( \T \right)$ are Banach Spaces. Are they related? 
     \end{question}
     \pause
      \end{frame}

 \begin{frame}{Poisson Kernel}

	 \begin{definition}[Poisson Kernel]
	 For each $r\in [0,1)$, we define $P_{r} : \T \to \R$ by
	 \begin{equation*}
	     P_{r} \left( e^{it} \right) = \frac{1-r^{2}}{1+r^{2}-2r\cos t}
	 \end{equation*}
 \end{definition}
 \pause
     \begin{itemize}
\item Recall the Dirichlet problem on the disk:
      Given $f: \T \to \C$. Does there exist a continuous function $u: \overline{\D} \to \C$ such that $u\mid _{\D}$  is harmonic and $u\mid_{\T} =f$?
  \pause
 \item  The solution to the Dirichlet problem on the $\D$ is:
  \begin{equation*}
      u\left( re^{i\theta} \right) =
      \begin{cases}
	  \frac{1}{2\pi} \int_{-\pi}^{\pi} P_{r} \left( e^{i\left( \theta-t \right)} \right) f\left( e^{it} \right) dt  & re^{i\theta} \in \D \\
	  f\left( e^{i\theta} \right) & e^{i\theta} \in \T
      \end{cases}
  \end{equation*}
\end{itemize}
\end{frame}
\begin{frame}{Poisson Integral}
     \only<3>{\begin{definition}[Poisson Integral]
	 Let $\mu \in \calM \left( \T \right)$ and $f\in L^{1}\left( \T \right)$. Then Poisson integral of $\mu$, denoted by $P\left[ \mu \right] : \D \to \C$ is given by
	 \begin{equation*}
	     P\left[ \mu \right] \left( re^{i\theta} \right) = \int_{\T} P_{r}\left( e^{i\left( \theta-t \right)} \right) d\mu \left( e^{it} \right)
	 \end{equation*}
	 and Poisson integral of $f$, denoted by $P\left[ f \right]: \D \to \C$ is given by
	 \begin{equation*}
	     P\left[ f \right] \left( re^{i\theta} \right) = \frac{1}{2\pi} \int_{-\pi}^{\pi} P_{r} \left( e^{i\left( \theta-t \right)} \right)f\left( e^{i\theta} \right)  dt.
	 \end{equation*}
     \end{definition}}
 \end{frame}

 %\begin{frame}{Representation Theorems}
 %    Let $u \in h\left( \D \right)$. Then $u$ is a Poisson integral of
 %    \begin{enumerate}
%	     \pause
%\item a function $f\in \calC \left( \T \right)$ iff $\lim_{r\nearrow 1} \norm{u_r-f}_{\infty} =0$.
	%     \pause
%	 \item a function $f\in L^{p}\left( \T \right)$, $1<p< \infty$ iff $\sup_{r\in [0,1)} \norm{u_{r}}_{p} < \infty$. In this case, $\norm{u_{r}-f}_{p} \to 0$.
%	 \item a complex measure $\mu$ iff $\sup_{r\in [0,1)} \norm{u_{r}}_{1}< \infty$. In this case, the measures $d\mu_{r} = \frac{1}{2\pi} u_{r}\left( e^{it} \right) dt$ converges to $d\mu$ in the weak-* topology on measures.
%	 \item a function $f\in L^{\infty} \left( \T \right)$ iff $\sup_{r\in [0,1)} \norm{u_{r}}_{\infty} < \infty$. In this case, $u_{r}$ converge to $f$ in the weak-* topology of $L^{\infty} \left( \T \right)$.
%	     \pause
%	 \item a positive measure iff $u$ is nonnegative.
 %    \end{enumerate}
 %    \pause
 %    The measures and functions are uniquely determined by $u$.
% \end{frame}

 \begin{frame}{Fatou's theorem (1906)}
     \begin{corollary}[\cite{fatou1906series}]
     Let $\mu \in \calM \left( \T \right)$. Then
     \begin{equation*}
	 \lim_{r\to 1} P[\mu]\left( re^{i\theta} \right)
     \end{equation*}
     exists for almost all $e^{i\theta} \in \T$ and equals "the" Radon Nikodym derivative of the absolutely continuous part of $\mu$ with respect to the Lebesgue measure. As a consequence, we have that if $f\in L^{1} \left( \T \right)$ then 
     \begin{equation*}
	 \lim_{r\to 1} P[f] \left( re^{i\theta} \right) = f\left( e^{i\theta} \right)
     \end{equation*}
     for almost all $e^{i\theta} \in \T$.
     \label{cor:Fatou}
 \end{corollary}
\end{frame}
\begin{frame}{Interaction of $\D$ and $\T$}
    Let $u: \D \to \C$ be a harmonic function and $1\le p \le \infty$. Suppose that for all $0\le r < 1$, we have that
    \begin{equation*}
	\norm{u_{r}}_{p} < M < +\infty
    \end{equation*}
    for some $M>0$.\pause Then for almost every $\theta$ the radial limits 
    \begin{equation*}
	\tilde {u} (e^{i\theta} ) = \lim_{r\to 1} u\left( re^{i\theta} \right)
    \end{equation*}
    exist and define a function $\tilde u$ in $L^{p} \left( \T \right)$. The following also holds: \pause
    \begin{enumerate}
	\item If $p>1$ then $u=P[\tilde{u}]$. \pause
	\item If $p=1$ then $f=P[\mu]$ for some complex measure $\mu$ whose absolutely continuous part is $\frac{1}{2\pi}\tilde{u}dt$.
\end{enumerate}
\end{frame}

\begin{frame}{Answering the Question}{$p>1$}
    For $p>1$, consider the map
    \begin{align*}
	H^{p} \left( \T \right) &\to H^{p} \left( \D \right) \\
	u &\mapsto P\left[ u \right]
    \end{align*}
    This is an isometric isomorphism.

\end{frame}
\begin{frame}{Answering the Question}{$p=1$}
    For $p=1$, consider the map
    \begin{align*}
	\calM_{a} \left( \T \right) &\to H^{1} \left( \D \right) \\
	\mu &\mapsto P\left[ \mu \right]
    \end{align*}
    This turns out to be a isometric isomorphism.

    \pause

    \begin{theorem}[F and M Riesz (1916)]
	Let $\mu \in \calM_{a} \left( \T \right)$ then $\mu$ is absolutely continuous.
\end{theorem}

\pause

\begin{align*}
    H^{1} \left( \T \right) &\to H^{1} \left( \D \right) \\
    f &\mapsto P[f]
\end{align*}
is again an isometric isomorphism.
\end{frame}
\section{\texorpdfstring{$H^{1}$}{H1} is nice!}
\begin{frame}{Szegő's theorem}{$H^{1}$}
    \begin{theorem}[Szegő]
	Let $f\in H^{1} \left( \T \right)$, $f\not\equiv 0$. Then the function $\log \abs{f}$ is integrable and 
	\begin{equation*}
	    \frac{1}{2\pi} \int_{\T} \log \abs{f\left( e^{it} \right)} dt \ge \log \abs{f\left( e^{i0} \right)}
	\end{equation*}
    \end{theorem}
     \pause
     \begin{corollary}
	 Let $f\in H^{1} \left( \T\right)$. If $f\not\equiv 0$ then $f$ cannot vanish on a (measurable) subset of $\T$ with positive Lebesgue measure.
     \end{corollary}
\end{frame}

\section{The Factorization}
\begin{frame}{Inner Function}
    \pause
	\begin{definition}[Inner Function]
    Let $f: \D \to \C$. Then $f$ is said to be an \textit{inner function} if $f\in H^{\infty} \left( \D \right)$ and the corresponding boundary function $\tilde{f} \in H^{\infty} \left( \T \right)$ has unit modulus almost everywhere on $\T$. In other words, the function $\tilde{f}$ defined almost everywhere on $\T$ by
    \begin{equation*}
	\tilde{f} \left( e^{i\theta} \right) = \lim_{r\to 1} f\left( re^{i\theta} \right)
    \end{equation*}
    has unit modulus almost everywhere.
\end{definition}
    
    \pause

    \begin{exampleblock}{Finite Blaschke Product}
    Let $z_{1}, z_{2}, \ldots, z_{n} \in \D$ and $\alpha \in \R$. Then the \text{finite Blaschke product} is the function given by 
    \begin{equation*}
	B\left( z \right) = e^{i\alpha} \prod_{k=1}^{n} \frac{z-z_{j}}{1-\bar{z_{j}}z}.
    \end{equation*}
\end{exampleblock}
\end{frame}

\begin{frame}{Outer Function}
    \begin{definition}[Outer Function]
	An \textit{outer function} is a holomorphic function $f: \D \to \C$ of the form 
	\begin{equation*}
	    f\left( re^{it} \right) = \alpha \exp\left[ \frac{1}{2\pi} \int_{-\pi}^{\pi} \frac{e^{it} + re^{i\theta}}{e^{it}-re^{i\theta}} k\left( e^{it} \right)\, dt \right]
	\end{equation*}
	where $\alpha \in \C$ with $\abs{\alpha}=1$ and $k$ is a real valued integrable function on $\T$.
    \end{definition}

    \pause

    \begin{proposition}
	Let $f: \D \to \C$ be an outer function with above form. Then 
	\begin{equation*}
	    f \in H^{1} \left( \D \right) \Leftrightarrow e^{k} \in L^{1} \left( \T \right)
	\end{equation*}
    \end{proposition}
\end{frame}

\begin{frame}{Inner Outer Factorization}
    \begin{theorem}[\cite{beurling1949two}]
	Let $f\in H^{1} \left( \D \right)$ and $f\not\equiv 0$. Then $f$ has a factorization $\theta \cdot u$ where $\theta$ is inner and $u$ is outer. This factorization is unique up to a constant of modulus $1$.
	\label{thm:inner-outer-not-complete}
    \end{theorem}
    \pause
    Idea of Proof: Define
    \begin{align*}
	u(z) &= \exp \left( \frac{1}{2\pi} \int_{-\pi}^{\pi} \frac{e^{it}+z}{e^{it}-z} \log \abs{\tilde{f}\left( e^{it} \right)} dt \right) \\
	\theta &= \frac{f}{u}.
    \end{align*}
\end{frame}

\begin{frame}{Disintegrating \textit{Inner} part}
    \begin{theorem}[nonzero $H^1$ functions satisfy the Blaschke condition]
	Let $f \in H^{1}\left( \D \right)$ and $f\not\equiv 0$. Then the zeroes of $f$ are countable in number and satisfy the \textbf{Blaschke condition}, that is, if $z_{1}, z_{2}, \ldots$ are the zeroes of $f$, then
    \begin{equation*}
	\sum_{k=1}^{\infty} \left( 1-\abs{z_{n}} \right) < \infty \Longleftrightarrow \prod_{n=1}^{\infty} \abs{z_{n}} < \infty
    \end{equation*}
    \end{theorem}
\end{frame}

\begin{frame}{Infinite Blaschke Products}
    \begin{theorem}[\cite{blaschke1915erweiterung}]
	Let $\left\{ z_{n} \right\}_{n\in \N} \subset \D \setminus \left\{ 0 \right\}$ be a sequence. The infinite product
    \begin{equation*}
	B\left( z \right) = \prod_{n=1}^{\infty} \frac{\bar{z_{n}}}{z_{n}} \frac{z_{n}-z}{1-\bar{z_{n}}z}
    \end{equation*}
    converges uniformly on compact subsets of $\D$ iff the product $\prod_{n=1}^{\infty} \abs{z_{n}}$ converges iff 
    \begin{equation*}
	\sum_{n=1}^{\infty} \left( 1-\abs{z_{n}} \right) < \infty.
    \end{equation*}
    When either of these is satisfied, $B$ defines an inner function whose zeroes are $\left\{ z_{n} : n \in \N \right\}$.
\end{theorem}
\end{frame}
\begin{frame}{Infinite Blaschke Products}
    \begin{definition}
	An \textit{(infinite) Blaschke product} is a holomorphic function $B$ of the form
	\begin{equation*}
	    B\left( z \right) = z^{p} \prod_{n=1}^{\infty} \left[ \frac{\bar{z_{n}}}{\abs{z_{n}}} \cdot \frac{z_{n}-z}{1-\bar{z_{n}}z} \right]^{p_{n}}
	\end{equation*}
	where
	\begin{enumerate}
	    \item $p,p_{1},p_{2}, \ldots  \in \N$;
	    \item $\left\{ z_{n} : n\in \N \right\} \subset \D \setminus \left\{ 0 \right\}$;
	    \item the product $\prod_{n=1}^{\infty} \abs{z_{n}}^{p_{n}}$ is convergent.
	\end{enumerate}
    \end{definition}

    \begin{corollary}[Factoring all zeroes of $H^{1} \left( \D \right)$]
	The Blaschke product formed out of the zeroes of a nonzero $H^{1} \left( \D \right)$ function is an inner function.
    \end{corollary}
\end{frame}

    \begin{frame}{So far\ldots}
	Let $f\in H^{1} \left( \D \right)$, $f\not \equiv 0$. Let
	\begin{equation*}
	    f=\theta \cdot u
	\end{equation*}
	be its inner outer factorisation where $\theta$ is inner and $u$ is outer.
	\pause
	The outer part $u$ has no zeroes of $F$, so, $\theta$ has all the zeroes.
	\pause
	Let $B$ be the Blaschke product formed out of zeroes of $u$ (which is same as that of $f$). Is
	\begin{equation*}
	    \theta /B
	\end{equation*}
	another inner function?
    \end{frame}
    \begin{frame}{Riesz Decomposition Theorem}
	\begin{theorem}[\cite{riesz1923randwerte}]
	    Let $f\in H^{p} \left( \D \right), 1\le p \le \infty$, $f\not \equiv 0$ and let $B$ be the Blaschke product formed with the zeroes of $f$ in $\D$. Let
	    \begin{equation*}
		g=f/B
	    \end{equation*}
	    Then $g\in H^{p} \left( \D \right)$, $g$ is zerofree in $\D$ and 
	    \begin{equation*}
		\norm{g}_{p} = \norm{f}_{p}.
	    \end{equation*}
	\end{theorem}
    \end{frame}

    \begin{frame}{Singular Inner Function}
	
\pause
Any nonzero $H^{1}$ function $f$ can be written as $B \cdot S \cdot u$ where $B$ is a Blaschke product, $S$ is a zerofree inner function and $u$ is the outer part of $f$. This representation is \textit{unique} upto multiplication by unimodular constant.

\pause

\begin{definition}[Singular Inner function]
    A inner function $S$ which is zerofree and $S\left( 0 \right) > 0$ is called singular function.
\end{definition}
\pause

\begin{theorem}[\cite{herglotz1911uber}]
    Let $g$ be a singular inner function. Then there is a unique singular positive measure $\mu$ such that 
    \begin{equation*}
	g\left( z \right) = \exp \left[ - \int_{\T} \frac{e^{it} + z}{e^{it} -z }d\mu \left( e^{it} \right) \right]
    \end{equation*}
\end{theorem}

    \end{frame}

    \begin{frame}{The Factorization Theorem}
    Let $f \not\equiv 0$ be an $H^{1}$ function in the unit disc. Then $f$ is uniquely expressible in the form of $f=B\cdot S \cdot u$ where $B$ is a Blaschke product, $S$ is a singular inner function and $u$ is an outer function (in $H^{1}$).	
    \pause

    Let $p$ be the order of zero of $f$ at the origin and let $p_{1}, p_{2}, \ldots$ be the multiplicities of the remaining zeroes $\alpha_{1}, \alpha_{2}, \ldots$ of $f$. 

Then we have that
\begin{align*}
    B\left( z \right) &= z^{p} \prod_{n=1}^{\infty} \left[ \frac{\overline{\alpha_{n}}}{\abs{\alpha_{n}}} \frac{\alpha_{n} - z}{ 1- \overline{\alpha_{n}}z} \right]^{p_{n}} \\
    u\left( z \right) &=  \exp \left[ \frac{1}{2\pi} \int_{-\pi}^{\pi} \frac{e^{i\theta} + z}{ e^{i\theta} -z} \left( \log \abs{\tilde{u}\left( e^{i\theta} \right)} + ia \right) d\theta \right]  \\
    S\left( z \right) &= \frac{f\left( z \right)}{B\left( z \right) u\left( z \right)} = \exp \left[ -\int \frac{e^{i\theta} +z}{ e^{i\theta} - z} d\mu \left( \theta \right) \right]
\end{align*}
for some positive singular measure $\mu$ and where $a= \arg \left( f/B \right) \left( 0 \right)$.
    \end{frame}

    \begin{frame}[allowframebreaks]{References}	\nocite{*}
    \printbibliography
\end{frame}

\begin{frame}
    \begin{center}
    \huge Thank You!
   \thispagestyle{empty} 
\end{center}
\end{frame}
\end{document}
