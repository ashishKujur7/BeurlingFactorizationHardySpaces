\documentclass{beamer}

\usepackage[english]{babel}
\usepackage{metalogo}
\usepackage{listings}
\usepackage{fontspec}
\usepackage{tikz}

\newtheorem{claim}[theorem]{Claim}
\newtheorem{proposition}[theorem]{Proposition}
\newtheorem{notation}[theorem]{Notation}
\newtheorem{observation}[theorem]{Observation}
\newtheorem{conjecture}[theorem]{Conjecture}
\newtheorem{exercise}[theorem]{Exercise}
\newtheorem{question}[theorem]{Question}

\numberwithin{equation}{subsection}


% section symbol
%\renewcommand{\thesection}{\S\arabic{section}}

% \renewcommand{\Pr}{{\bf Pr}}
% \newcommand{\Prx}{\mathop{\bf Pr\/}}
% \newcommand{\E}{{\bf E}}
% \newcommand{\Ex}{\mathop{\bf E\/}}
% \newcommand{\Var}{{\bf Var}}
% \newcommand{\Varx}{\mathop{\bf Var\/}}
% \newcommand{\Cov}{{\bf Cov}}
% \newcommand{\Covx}{\mathop{\bf Cov\/}}

% shortcuts for symbol names that are too long to type
\newcommand{\eps}{\epsilon}
\newcommand{\lam}{\lambda}
\renewcommand{\l}{\ell}
\newcommand{\la}{\langle}
\newcommand{\ra}{\rangle}
\newcommand{\wh}{\widehat}
\newcommand{\wt}{\widetilde}

% % "blackboard-fonted" letters for the reals, naturals etc.
\newcommand{\R}{\mathbb R}
\newcommand{\N}{\mathbb N}
\newcommand{\Z}{\mathbb Z}
\newcommand{\F}{\mathbb F}
\newcommand{\Q}{\mathbb Q}
\newcommand{\C}{\mathbb C}
\newcommand{\D}{\mathbb D}
\newcommand{\T}{\mathbb T}

% % operators that should be typeset in Roman font
% \newcommand{\poly}{\mathrm{poly}}
% \newcommand{\polylog}{\mathrm{polylog}}
% \newcommand{\sgn}{\mathrm{sgn}}
% \newcommand{\avg}{\mathop{\mathrm{avg}}}
% \newcommand{\val}{{\mathrm{val}}}

% % complexity classes
% \renewcommand{\P}{\mathrm{P}}
% \newcommand{\NP}{\mathrm{NP}}
% \newcommand{\BPP}{\mathrm{BPP}}
% \newcommand{\DTIME}{\mathrm{DTIME}}
% \newcommand{\ZPTIME}{\mathrm{ZPTIME}}
% \newcommand{\BPTIME}{\mathrm{BPTIME}}
% \newcommand{\NTIME}{\mathrm{NTIME}}

% values associated to optimization algorithm instances
\newcommand{\Opt}{{\mathsf{Opt}}}
\newcommand{\Alg}{{\mathsf{Alg}}}
\newcommand{\Lp}{{\mathsf{Lp}}}
\newcommand{\Sdp}{{\mathsf{Sdp}}}
\newcommand{\Exp}{{\mathsf{Exp}}}

% if you think the sum and product signs are too big in your math mode; x convention
% as in the probability operators
\newcommand{\littlesum}{{\textstyle \sum}}
\newcommand{\littlesumx}{\mathop{{\textstyle \sum}}}
\newcommand{\littleprod}{{\textstyle \prod}}
\newcommand{\littleprodx}{\mathop{{\textstyle \prod}}}


% calligraphic letters
\newcommand{\calA}{{\cal A}}
\newcommand{\calB}{{\cal B}}
\newcommand{\calC}{{\cal C}}
\newcommand{\calD}{{\cal D}}
\newcommand{\calE}{{\cal E}}
\newcommand{\calF}{{\cal F}}
\newcommand{\calG}{{\cal G}}
\newcommand{\calH}{{\cal H}}
\newcommand{\calI}{{\cal I}}
\newcommand{\calJ}{{\cal J}}
\newcommand{\calK}{{\cal K}}
\newcommand{\calL}{{\cal L}}
\newcommand{\calM}{{\cal M}}
\newcommand{\calN}{{\cal N}}
\newcommand{\calO}{{\cal O}}
\newcommand{\calP}{{\cal P}}
\newcommand{\calQ}{{\cal Q}}
\newcommand{\calR}{{\cal R}}
\newcommand{\calS}{{\cal S}}
\newcommand{\calT}{{\cal T}}
\newcommand{\calU}{{\cal U}}
\newcommand{\calV}{{\cal V}}
\newcommand{\calW}{{\cal W}}
\newcommand{\calX}{{\cal X}}
\newcommand{\calY}{{\cal Y}}
\newcommand{\calZ}{{\cal Z}}

% proof
%\renewcommand{\qedsymbol}{$\ddot\smile$}


% bold letters (useful for random variables)
%----------------------------------------------
% \renewcommand{\a}{{\boldsymbol a}}
% \renewcommand{\b}{{\boldsymbol b}}
% \renewcommand{\c}{{\boldsymbol c}}
% \renewcommand{\d}{{\boldsymbol d}}
% \newcommand{\e}{{\boldsymbol e}}
% \newcommand{\f}{{\boldsymbol f}}
% \newcommand{\g}{{\boldsymbol g}}
% \newcommand{\h}{{\boldsymbol h}}
% \renewcommand{\i}{{\boldsymbol i}}
% \renewcommand{\j}{{\boldsymbol j}}
% \renewcommand{\k}{{\boldsymbol k}}
% \newcommand{\m}{{\boldsymbol m}}
% \newcommand{\n}{{\boldsymbol n}}
% \renewcommand{\o}{{\boldsymbol o}}
% \newcommand{\p}{{\boldsymbol p}}
% \newcommand{\q}{{\boldsymbol q}}
% \renewcommand{\r}{{\boldsymbol r}}
% \newcommand{\s}{{\boldsymbol s}}
% \renewcommand{\t}{{\boldsymbol t}}
% \renewcommand{\u}{{\boldsymbol u}}
% \renewcommand{\v}{{\boldsymbol v}}
% \newcommand{\w}{{\boldsymbol w}}
% \newcommand{\x}{{\boldsymbol x}}
% \newcommand{\y}{{\boldsymbol y}}
% \newcommand{\z}{{\boldsymbol z}}
% \newcommand{\A}{{\boldsymbol A}}
% \newcommand{\B}{{\boldsymbol B}}
% \newcommand{\C}{{\boldsymbol C}}
% \newcommand{\D}{{\boldsymbol D}}
% \newcommand{\E}{{\boldsymbol E}}
% \newcommand{\F}{{\boldsymbol F}}
% \newcommand{\G}{{\boldsymbol G}}
% \renewcommand{\H}{{\boldsymbol H}}
% \newcommand{\I}{{\boldsymbol I}}
% \newcommand{\J}{{\boldsymbol J}}
% \newcommand{\K}{{\boldsymbol K}}
% \renewcommand{\L}{{\boldsymbol L}}
% \newcommand{\M}{{\boldsymbol M}}
% \renewcommand{\O}{{\boldsymbol O}}
% \renewcommand{\P}{{\mathbb{P}}}
% \newcommand{\Q}{{\boldsymbol Q}}
% \newcommand{\R}{{\boldsymbol R}}
% \renewcommand{\S}{{\boldsymbol S}}
% \newcommand{\T}{{\boldsymbol T}}
% \newcommand{\U}{{\boldsymbol U}}
% \newcommand{\V}{{\boldsymbol V}}
% \newcommand{\W}{{\boldsymbol W}}
% \newcommand{\X}{{\boldsymbol X}}
% \newcommand{\Y}{{\boldsymbol Y}}
% \newcommand{\Z}{{\boldsymbol Z}}

% script letters
\newcommand{\scrA}{{\mathscr A}}
\newcommand{\scrB}{{\mathscr B}}
\newcommand{\scrC}{{\mathscr C}}
\newcommand{\scrD}{{\mathscr D}}
\newcommand{\scrE}{{\mathscr E}}
\newcommand{\scrF}{{\mathscr F}}
\newcommand{\scrG}{{\mathscr G}}
\newcommand{\scrH}{{\mathscr H}}
\newcommand{\scrI}{{\mathscr I}}
\newcommand{\scrJ}{{\mathscr J}}
\newcommand{\scrK}{{\mathscr K}}
\newcommand{\scrL}{{\mathscr L}}
\newcommand{\scrM}{{\mathscr M}}
\newcommand{\scrN}{{\mathscr N}}
\newcommand{\scrO}{{\mathscr O}}
\newcommand{\scrP}{{\mathscr P}}
\newcommand{\scrQ}{{\mathscr Q}}
\newcommand{\scrR}{{\mathscr R}}
\newcommand{\scrS}{{\mathscr S}}
\newcommand{\scrT}{{\mathscr T}}
\newcommand{\scrU}{{\mathscr U}}
\newcommand{\scrV}{{\mathscr V}}
\newcommand{\scrW}{{\mathscr W}}
\newcommand{\scrX}{{\mathscr X}}
\newcommand{\scrY}{{\mathscr Y}}
\newcommand{\scrZ}{{\mathscr Z}}

\newcommand{\im}{{\text{im }}}
\newcommand{\ip}[1]{\left\langle #1 \right\rangle}
\newcommand{\norm}[1]{\left\lVert #1 \right\rVert}
\newcommand{\abs}[1]{\left\lvert #1 \right\rvert}


\usetheme{Nord}
% \usetheme[style=light]{Nord}

%\setmainfont{JetBrains Mono Regular}
%\setsansfont{JetBrains Mono Regular}
%\setmonofont{JetBrains Mono Regular}

%\setmainfont{Noto Sans Regular}
%\setsansfont{Noto Sans Regular}
%\setmonofont{Noto Sans Mono Regular}

\AtBeginSection[]
{
  \begin{frame}[c,noframenumbering,plain]
    \tableofcontents[sectionstyle=show/hide,subsectionstyle=show/show/hide]
  \end{frame}
}

\AtBeginSubsection[]
{
  \begin{frame}[c,noframenumbering,plain]
    \tableofcontents[sectionstyle=show/hide,subsectionstyle=show/shaded/hide]
  \end{frame}
}

\title{Beurling Factorisation of Hardy Spaces}
\subtitle{Final Presentation}
\author{Ashish Kujur}
%\institute{Indian Institute of Science Education and Research}
\date{16th May, 2023}

\begin{document}
 \begin{frame}[plain,noframenumbering]
   \maketitle
 \end{frame}


 \begin{frame}{Notation and Conventions}
     \begin{itemize}
	     \pause
	 \item (Open Unit Disc) $\D = \left\{ z\in \C : \abs{z} < 1 \right\}$.
	     \pause
	 \item (Unit Circle) $\T = \left\{ z\in \C : \abs{z}=1 \right\}$.
	     \pause
	 \item $\T$ has the normalised Lebesgue measure $\frac{dt}{2\pi}$ unless specified otherwise.
\pause
	 \item $\calM \left( \T \right)$: Banach space of complex measures on $\T$ with the total variation norm.
	     \pause
	 \item \textit{nth Fourier coefficient} of $f\in \calL ^{1}\left( \T \right)$ and $\mu \in \calM \left( \T \right)$, $n\in \Z$:
	     \begin{equation*}
		 \hat{f}(n) := \frac{1}{2\pi} \int_{-\pi}^{\pi} f\left( e^{it} \right) e^{-int} dt.
	     \end{equation*}
	     \begin{equation*}
		 \hat{\mu} \left( n \right) := \int_{\T} e^{-int} d\mu \left( e^{it} \right).
	     \end{equation*}
	     \pause
	 \item $H\left( \D \right) = \left\{ f : \D \to \C : f \text{ is holomorphic on } \D \right\}$ \\
	     $h\left( \D \right) = \left\{ f: \D \to \C : f \text{ is harmonic in } \D \right\}$.
     \end{itemize}
 \end{frame}

 \begin{frame}{Hardy Spaces on $\D$}{(F. Riesz (1923))}
     \pause
     Let $1\le p \le \infty$ and $f\in H \left( \D \right)$.

\pause 
For $0\le r < 1$, define $f_{r} : \T \to \C$, $f_{r} \left( e^{i\theta} \right) = f\left( re^{i\theta} \right)$ for each $e^{i\theta} \in \T$.
     \pause

     Define $H^{p} \left( \D \right)$, \textit{Hardy class of analytic functions}, by 
     \begin{equation*}
	 H^{p} \left( \D \right) = \left\{ f\in H\left( \D \right) : \left\{ \norm{f_{r}}_{p} \right\}_{0\le r < 1} \text{ is bounded} \right\} 
     \end{equation*}
\pause
\begin{theorem}[G.H. Hardy, 1915]
	 If $f\in H^{p} \left( \D \right)$ then
	     $\norm{f_{r_{1}}}_{p} \le \norm{f_{r_{2}}}_{p}$
	 for $0 < r_{1} \le r_{2} < 1$.
	 \label{thm:increasing-norm}
     \end{theorem}
\pause
     \begin{theorem}
	 For $1\le p < \infty$, $H^{p} \left( \D \right)$ is a Banach space with the norm
	 \begin{equation*}
	     \norm{f}_{p}:=\sup_{0< r < 1} \norm{f_r}_{p} = \lim_{r \to 1} \norm{f_{r}}_{p}.
	 \end{equation*}
	 \label{thm:HpD-Banach}
     \end{theorem}
 \end{frame}

 \begin{frame}{Hardy Spaces on $\T$}
     \pause
     Let $1 \le p \le \infty$. Consider the measure space $\left( \T, \calB\left( \T \right), dt/2\pi \right)$. 
     \pause
     Define
     $H^{p} \left( \T \right) = \left\{ f\in L^{p} \left( \T \right) : \hat{f}\left( n \right) = 0 \text{ for each } n<0 \right\}$.
     
     \pause
     $H^{p}\left( \T \right)$ is a Banach space. \pause (Why?)

     \pause
     \begin{question}
	 $H^{p} \left( \D \right)$ and $H^{p}\left( \T \right)$ are both Banach spaces. Are they related?
     \end{question}
     \pause
     \begin{center}
     YES!
 \end{center}
 \end{frame}

 \begin{frame}{Poisson Kernel \& Integral}{(Recover!)}
     \only<1>{\begin{definition}[Poisson Kernel]
	 For each $r\in [0,1)$, we define $P_{r} : \T \to \R$ by
	 \begin{equation*}
	     P_{r} \left( e^{it} \right) = \frac{1-r^{2}}{1+r^{2}-2r\cos t}
	 \end{equation*}
 \end{definition}}
     
     \only<2>{\begin{definition}[Poisson Integral]
	 Let $\mu \in \calM \left( \T \right)$ and $f\in L^{1}\left( \T \right)$. Then Poisson integral of $\mu$, denoted by $P\left[ \mu \right] : \D \to \C$ is given by
	 \begin{equation*}
	     P\left[ \mu \right] \left( re^{i\theta} \right) = \int_{\T} P_{r}\left( e^{i\left( \theta-t \right)} \right) d\mu \left( e^{it} \right)
	 \end{equation*}
	 and Poisson integral of $f$, denoted by $P\left[ f \right]: \D \to \C$ is given by
	 \begin{equation*}
	     P\left[ f \right] \left( re^{i\theta} \right) = \frac{1}{2\pi} \int_{-\pi}^{\pi} P_{r} \left( e^{i\left( \theta-t \right)} \right)f\left( e^{i\theta} \right)  dt.
	 \end{equation*}

     \label{def:Poisson-Integral-Measure}
     \end{definition}}
 \end{frame}

 \begin{frame}{Representation Theorems}{(Recovery)}
     Let $u \in h\left( \D \right)$. Then $u$ is a Poisson integral of
     \begin{enumerate}
	 \item hello
     \end{enumerate}
 \end{frame}
\end{document}
